% Created 2019-05-08 Wed 20:00
% Intended LaTeX compiler: pdflatex
\documentclass[12pt, a4paper]{report}
\usepackage[utf8]{inputenc}
\usepackage[T1]{fontenc}
\usepackage[czech, ]{babel}
\usepackage{graphicx}
\usepackage{grffile}
\usepackage{longtable}
\usepackage{float}
\usepackage{wrapfig}
\usepackage{rotating}
\usepackage[normalem]{ulem}
\usepackage{amsmath}
\usepackage{textcomp}
\usepackage{amssymb}
\usepackage{capt-of}
\usepackage[hidelinks]{hyperref}
\usepackage{csquotes}
\usepackage{tabularx}
\MakeOuterQuote{"}
\usepackage{setspace}
\onehalfspacing
\usepackage{titlesec}
\titleformat{\chapter}[display]{\normalfont\bfseries}{}{0pt}{\Huge}
\usepackage[backend=bibtex,citestyle=authoryear]{biblatex}
\addbibresource{~/OneDrive/Orgmode/Papers/references.bib}


\DeclareUnicodeCharacter{00A0}{~}
\usepackage{tikz}
\usetikzlibrary{mindmap}
% \addbibresource{./references.bib} % pro pouziti lokalni slozky se zdroji
\author{Matěj Haša, Vít Chrubasík, Marek Štěpán}
\date{Květen 2019}
\title{Projekt: Státní maturita s Khan Academy\\\medskip
\large \includegraphics[width=\linewidth]{./images/skola_logo.png} Seminární práce}
\hypersetup{
 pdfauthor={Matěj Haša, Vít Chrubasík, Marek Štěpán},
 pdftitle={Projekt: Státní maturita s Khan Academy},
 pdfkeywords={},
 pdfsubject={},
 pdfcreator={Emacs 26.1 (Org mode 9.2.3)}, 
 pdflang={Czech}}
\begin{document}

\maketitle
\tableofcontents
\thispagestyle{empty}



\chapter{Úvod}
\label{sec:org363e3db}
V této seminární práci se budeme snažit co nejvěrněji uvést do praxe nabyté vědomosti, získané z hodin kurzu Projektové řízení A, které se týkají úvodní fáze projektu. V této fázi se především projekt plánuje, analyzuje a ověřuje se, jestli je proveditelný v kontextu našich možných zdrojů, efektivní etc. Z tohoto důvodu jsme práci rozdělili do několika kapitol, jak můžete vidět v obsahu v následující sekci. 

Náš fiktivní projekt má za cíl implementaci výukových materiálů pro českou státní maturitu do již existující online vzdělávací platformy Khan Academy. Jak už název napovídá, jedná se o online výukovou službu, dostupnou na internetové síti. V aktuálním řešení (Květen 2019) jsou dostupné kurzy z většiny vědních oborů jako např. matematiky, biologie, chemie, fyziky, historie etc. Khan Academy je dostupná i v českém jazyce. S kolegy bychom chtěli vytvořit projekt, jehož obsahem bude rozšíření výukových materiálů na verzi s českou jazykovou mutací a to o výukové materiály pro českou státní maturitu.
\chapter{Identifikační listina projektu}
\label{sec:orgb803dc7}
\chapter{Identifikační listina}
\label{sec:org08a7d98}
\chapter{Logický rámec}
\label{sec:orgb9e6b0d}
\chapter{SWOT analýza}
\label{sec:org551b973}
\chapter{Myšlenková mapa}
\label{sec:org7efa980}
\begin{tikzpicture}[mindmap, grow cyclic, every node/.style=concept, concept color=blue!40, 
	level 1/.append style={level distance=4cm,sibling angle=60},
	level 2/.append style={level distance=3cm,sibling angle=45},]

\node{Projekt}
child { node {Najít učitele/lektory} [concept color=red!60]
	child { node {Neschopní, neochotní pracovníci}}
}
child { node {Technická implementace} [clockwise from=45, concept color=red!60]
	child { node {Špatné testování}}
}
child { node {Statistické vyhodnocení úspěšnosti} [clockwise from=45, concept color=red!60]
	child { node {Statisticky nehodnotný vzorek}}
}
child { node {Vytvoření osnovy} [concept color=red!60]
	child { node {Změna osnov maturity}}
}
child { node {Domluva s Khan Academy} [concept color=red!60]
	child { node {Neshoda}}
}
child { node {Vypracování materiálů} [concept color=red!60]
	child { node {Chyby v materiálech}}
};

\end{tikzpicture}


\chapter{MS Project}
\label{sec:org3db359f}
\chapter{Budoucí vývoj}
\label{sec:orgf5bae8a}
\chapter{Závěr}
\label{sec:org338994c}
\end{document}