% Created 2019-05-09 Thu 14:15
% Intended LaTeX compiler: pdflatex
\documentclass[12pt, a4paper]{report}
\usepackage[utf8]{inputenc}
\usepackage[T1]{fontenc}
\usepackage[czech, ]{babel}
\usepackage{graphicx}
\usepackage{grffile}
\usepackage{longtable}
\usepackage{float}
\usepackage{wrapfig}
\usepackage{rotating}
\usepackage[normalem]{ulem}
\usepackage{amsmath}
\usepackage{textcomp}
\usepackage{amssymb}
\usepackage{capt-of}
\usepackage[hidelinks]{hyperref}
\usepackage{csquotes}
\usepackage{tabularx}
\MakeOuterQuote{"}
\usepackage{setspace}
\onehalfspacing
\usepackage{titlesec}
\titleformat{\chapter}[display]{\normalfont\bfseries}{}{0pt}{\Huge}
\usepackage[backend=bibtex,citestyle=authoryear]{biblatex}
\addbibresource{~/OneDrive/Orgmode/Papers/references.bib}


\DeclareUnicodeCharacter{00A0}{~}
\usepackage{tikz}
\usetikzlibrary{mindmap}
% \addbibresource{./references.bib} % pro pouziti lokalni slozky se zdroji
\author{Matěj Haša, Vít Chrubasík, Marek Štěpán}
\date{Květen 2019}
\title{Projekt: Státní maturita s Khan Academy\\\medskip
\large \includegraphics[width=\linewidth]{./images/skola_logo.png} Seminární práce}
\hypersetup{
 pdfauthor={Matěj Haša, Vít Chrubasík, Marek Štěpán},
 pdftitle={Projekt: Státní maturita s Khan Academy},
 pdfkeywords={},
 pdfsubject={},
 pdfcreator={Emacs 26.1 (Org mode 9.2.3)}, 
 pdflang={Czech}}
\begin{document}

\maketitle
\tableofcontents
\thispagestyle{empty}



\chapter{Úvod}
\label{sec:org7a1d734}
V této seminární práci se budeme snažit co nejlépe aplikovat nabyté vědomosti, získané z hodin kurzu Projektové řízení A, které se týkají úvodní fáze projektu. V této fázi se především projekt plánuje, analyzuje a ověřuje se, jestli je proveditelný v kontextu našich možných zdrojů, efektivní etc. Z tohoto důvodu jsme práci rozdělili do několika kapitol, jak můžete vidět v obsahu v následující sekci. 

Náš fiktivní projekt má za cíl implementaci výukových materiálů pro českou státní maturitu do již existující online vzdělávací platformy Khan Academy. Jak už název napovídá, jedná se o online výukovou službu, dostupnou na internetové síti. V aktuálním řešení (Květen 2019) jsou dostupné kurzy z většiny vědních oborů jako např. matematiky, biologie, chemie, fyziky, historie etc. Khan Academy je dostupná i v českém jazyce. S kolegy bychom chtěli vytvořit projekt, jehož obsahem bude rozšíření výukových materiálů na verzi s českou jazykovou mutací a to o výukové materiály pro českou státní maturitu z matematiky.
\chapter{Identifikační listina}
\label{sec:org2fe0249}
\chapter{Logický rámec}
\label{sec:orgd1ecb1c}

\begin{table}[htbp]
\caption{\textbf{Logický rámec}}
\centering
\begin{tabularx}{\textwidth}{XXXX}
Sloupec intervencí & Objektivně měřitelný ukazatel & Zdroje informací k ověření & Rizika / předpoklady (vnější)\\
\hline
Zvýšit úspěšnost u státní maturity z matematiky & Absolventi kurzu & Dotazníky, výsledky absolventů & -\\
\hline
Poskytnutí možnosti dobré přípravy ke zkoušce & Absolventi kurzu & Statistické údaje & Správně sestavený kurz, který se osvědčil v dobré přípravě studentů\\
\hline
Sekce Maturita na na české Khan Academy & Beta-testování, testovací a  kontrolní skupina & reporty z testování & Validní testovací scénáře, úspěšná fáze testování, statisticky významné zlepšení testovací skupiny\\
\hline
Návrh, implementace, testování, nasazení, údržba & Lidská práce, know-how, hardware, technologie, energie & Vedoucí pracovníci, road map, účetní výkazy & Správná analýza, kvalitní vývojová platforma, dobře fungující tým\\
\hline
 &  &  & Svolení k realizaci projektu ze strany Khan Academy\\
\end{tabularx}
\end{table}

\chapter{analýza SWOT}
\label{sec:org2eb4c79}
\begin{table}[htbp]
\caption{SWOT analýza}
\centering
\begin{tabularx}{\textwidth}{Xrrr}
Položka & Váha & Hodnocení & Hodnota\\
\hline
\textbf{Silné stránky} &  &  & \\
Materiálová nenáročnost & 10 & 2 & 0.2\\
Veřejně prospěšná činnost & 20 & 3 & 0.6\\
Originalita & 30 & 5 & 1.5\\
Implementace do zaběhnutého systému & 40 & 5 & 2.\\
\hline
\textbf{Silné stránky celkem} & 100 &  & 4.3\\
\hline
\textbf{Slabé stránky} &  &  & \\
Malý tým & 12 & 4 & -0.48\\
Nově sestavený tým & 20 & 1 & -0.2\\
Nizký rozpočet & 33 & 2 & -0.66\\
Málo praktických zkušeností & 35 & 3 & -1.05\\
\hline
\textbf{Slabé stránky celkem} & 100 &  & -2.39\\
\hline
\textbf{Příležitosti} &  &  & \\
Dotace & 25 & 2 & 0.5\\
Dobrovolníci & 47 & 4 & 1.88\\
Zvýšení poptávky & 12 & 1 & 0.12\\
Nefinanční příspěvky & 16 & 3 & 0.48\\
\hline
\textbf{Příležitosti celkem} & 100 &  & 2.98\\
\hline
\textbf{Hrozby} &  &  & \\
Změní se osnovy maturity & 35 & 3 & -1.05\\
Konkurence & 6 & 1 & -0.06\\
Odmítnutí návrhu implementace & 40 & 3 & -1.2\\
Zrušení státní maturity & 19 & 1 & -0.19\\
\hline
\textbf{Hrozby celkem} & 100 &  & -2.5\\
\hline
\textbf{Interní faktory} &  &  & 1.91\\
\textbf{Externí faktory} &  &  & 0.48\\
\hline
\textbf{Závěr} &  &  & \textbf{2.39}\\
\end{tabularx}
\end{table}

\chapter{Myšlenková mapa}
\label{sec:orgaeadcd5}
\begin{tikzpicture}[mindmap, grow cyclic, every node/.style=concept, concept color=blue!40, 
	level 1/.append style={level distance=4cm,sibling angle=60},
	level 2/.append style={level distance=3cm,sibling angle=45},]

\node{Projekt}
child { node {Najít učitele/lektory} [concept color=red!60]
	child { node {Neschopní, neochotní pracovníci}}
}
child { node {Technická implementace} [clockwise from=45, concept color=red!60]
	child { node {Špatné testování}}
}
child { node {Statistické vyhodnocení úspěšnosti} [clockwise from=45, concept color=red!60]
	child { node {Statisticky nehodnotný vzorek}}
}
child { node {Vytvoření osnovy} [concept color=red!60]
	child { node {Změna osnov maturity}}
}
child { node {Domluva s Khan Academy} [concept color=red!60]
	child { node {Neshoda}}
}
child { node {Vypracování materiálů} [concept color=red!60]
	child { node {Chyby v materiálech}}
};

\end{tikzpicture}



\chapter{MS Project}
\label{sec:orgf8f5992}
\chapter{Budoucí vývoj}
\label{sec:orgde1ea73}
\chapter{Závěr}
\label{sec:org78226fb}
\end{document}